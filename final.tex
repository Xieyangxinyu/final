\documentclass[10pt, oneside]{article}
\usepackage{geometry}             
\geometry{letterpaper, left = 21 mm, right = 21 mm, top = 30 mm, bottom = 25 mm}     


%------------------------macros and packages------------------

%%%Please put path to the downloaded macro file here.
\usepackage{macros} 
%-------------------------begin doc-------------------


\begin{document}

\thispagestyle{empty}
\title{Combinatorial Hypothesis Testing}


%-------------------------------------------------------------------------%

\maketitle
\tableofcontents
\let\thefootnote\relax\footnote{Last updated \today}
\addtocounter{footnote}{-1}\let\thefootnote\svthefootnote

\section{Introduction}

Suppose we observe an $n$-dimensional vector $\bX = (X_1,...,X_n)$. The null hypothesis $H_0$ is that the components of $\bX$ are independent and identically distributed (i.i.d.) standard normal random variables. We denote the probability measure and expectation under $H_0$ by $\bP_0$ and $\bE_0$, respectively.

Combinatorics kicks in as we consider the alternative hypotheses: consider a class $\cC=\{S_1,\ldots,S_N\}$ of $N$ sets of indices such that $S_k \subset\{1,\ldots,n\}$ for all $k=1,\ldots,N$. Under $H_1$, there
exists an $S \in \cC$ such that $X_i$ has a distribution determined by whether $i$ is in $S$:
\begin{enumerate}
  \item In its simplest form, as discussed in \cite{arias2008searching, addario2010combinatorial, arias2011detection}, we consider 
$$
X_i \mbox{ has distribution }
\begin{cases}
  \cN(0,1), \quad & \mbox{ if }i \notin S\\
  \cN(\mu,1), & \mbox{ if }i \in S
\end{cases}
$$
where $\mu>0$ is a positive parameter and components of $\bX$ are independent. 
  \item 
\end{enumerate}

\section{Moment Methods}

\section{Extension}

\bibliographystyle{plain}
\bibliography{bibfile}
\end{document}
