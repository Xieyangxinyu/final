\documentclass[10pt, oneside]{article}
\usepackage{geometry}             
\geometry{letterpaper, left = 21 mm, right = 21 mm, top = 30 mm, bottom = 25 mm}     


%------------------------macros and packages------------------

%%%Please put path to the downloaded macro file here.
\usepackage{macros} 
%-------------------------begin doc-------------------


\begin{document}

\thispagestyle{empty}
\title{Combinatorial Hypothesis Testing}


%-------------------------------------------------------------------------%

\maketitle
\addtocounter{footnote}{-1}\let\thefootnote\svthefootnote

\section{Introduction}

Suppose we observe an $n$-dimensional vector $\bX = (X_1,...,X_n)$. The null hypothesis $H_0$ is that the components of $\bX$ are independent and identically distributed (i.i.d.) standard normal random variables. We denote the probability measure and expectation under $H_0$ by $\bP_0$ and $\bE_0$, respectively.

Combinatorics kicks in as we consider the alternative hypotheses, by introducing a class $\cC$ with some combinatorial structure: consider a class $\cC=\{S_1,\ldots,S_N\}$ of $N$ sets of indices such that $S_k \subset\{1,\ldots,n\}$ for all $k=1,\ldots,N$. Under $H_1$, there
exists an $S \in \cC$ such that $X_i$ has a distribution determined by whether $i$ is in $S$:
\begin{enumerate}
  \item In its simplest form, as discussed in \cite{arias2008searching, addario2010combinatorial, arias2011detection}, we consider 
$$
X_i \mbox{ has distribution }
\begin{cases}
  \cN(0,1), \quad & \mbox{ if }i \notin S\\
  \cN(\mu,1), & \mbox{ if }i \in S
\end{cases}
$$
where $\mu>0$ is a positive parameter and components of $\bX$ are independent. 
  \item In testing correlations \cite{arias2012correlation}, we consider $$
  \Cov(X_i, X_j) =
  \begin{cases}
    1, \quad & \mbox{ if }i = j\\
    \rho, & \mbox{ if }i \neq j\mbox{ with }i, j \in S\\
    0, & \mbox{ otherwise}\\
  \end{cases}
  $$
\end{enumerate}

For each $S \in \cC$, we denote the probability measure and expectation by $\bP_S$ and $\bE_S$, respectively. Many interesting examples of $\cC$ arises for this scenario: subsets of size $K$, cliques, perfect matchings, spanning trees, and clusters.

A~\textit{test} is a~binary-valued function $f:\bR^n \to\{
0,1\}$. If
$f(X)=0$, then the test accepts the null hypothesis $H_0$;
otherwise $H_0$ is rejected by $f$.
We measure the performance of a~test based on the \textit{minimax risk}:
\[
R_*^{\max} := \inf_{f} R^{\max}(f).
\]
where $R^{\max}(f)$ is the worst-case risk over the class of interest $\cC$, formally defined by
\[
R^{\max}(f) = \bP_0\{f(X)=1\}
+ \max_{S\in\cC} \bP_S\{f(X)=0\}.
\]

In this report, we discuss the techniques introduced in \cite{arias2012correlation, addario2010combinatorial, arias2011detection} to derive the asymptotic upper and lower bounds of $R_*^{\max}$, as well as more recent extensions.
\section{Lower Bounds}
\subsection{Moment Methods}
\label{subsec:Moment Methods}
A~standard way of obtaining lower bounds for the minimax risk
is by putting a~prior on the
class $\cC$ and obtaining a~lower bound on the corresponding \textit
{Bayesian risk}, which never exceeds the worst-case risk. Because
this is true for any prior, the idea is to find one that is hardest
(often called \textit{least favorable}). Consider the
uniform prior on $\cC$, giving rise to the following \textit{average risk}:
%
\[
R(f) = \bP_0\{f(X)=1\}
+ \bP_1\{f(X)=0\},
\]
%
where
%
\[
\bP_1\{f(X)=0\} := \frac{1}{N}\sum_{S\in\cC} \bP_S\{f(X)=0\},
\]
%
and $N := |\cC|$ is the cardinality of $\cC$.
The advantage of considering the average risk over the worst-case
risk is that we know an optimal test for the former, which, by the
Neyman--Pearson fundamental lemma, is the likelihood ratio test,
denoted $f^*$. Introducing $L(X)$, the likelihood ratio between $H_0$ and $H_1$, the optimal test becomes
%
\[
f^*(x) = 0  \quad\mbox{if and only if}\quad   L(x) \le 1.
\]
The
%GL
(average)
risk $R^*=R(f^*)$ of the optimal test is called the
\textit{Bayes risk} and it satisfies

\[
R^* = 1 - \frac{1}{2} \bE_0 |L(X) - 1|
\]



\section{Clusters}
\label{subsec:Clusters}
\section{Extension}

\bibliographystyle{plain}
\bibliography{bibfile}
\end{document}
